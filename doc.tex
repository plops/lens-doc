% first i tried to use org mode, but that doesn't have sufficient
% support for formulas and images
\documentclass[11pt]{article}
\usepackage[utf8]{inputenc}
\usepackage[T1]{fontenc}
\usepackage{graphicx}
\usepackage{longtable}
\usepackage{float}
\usepackage{wrapfig}
\usepackage{soul}
\usepackage{amssymb}
\usepackage{amsmath}
\usepackage{hyperref}
\usepackage{color}

\title{aberration}
\author{martin}
\date{31 March 2011}

\begin{document}

\maketitle

\setcounter{tocdepth}{3}
\tableofcontents
\vspace*{1cm}

\newcommand{\vect}[1]{\mathbf{#1}}
\renewcommand{\r}{\vect r}
\renewcommand{\a}{\vect a}
\newcommand{\s}{\vect s}
\renewcommand{\k}{\vect k}

\newcommand{\nvect}[1]{\vect{\hat{#1}}}
\renewcommand{\i}{\nvect i}
\newcommand{\n}{\nvect n}
\newcommand{\vrho}{\vect\rho}
\newcommand{\bild}[1]{\includegraphics[width=7cm]{#1}}
\section{Aberration due to water}
\subsection{Refraction on thin lens}
\bild{thin-lens.jpg}

\begin{align}
  \r'&=\i- \frac{\cos\phi}{f} \vrho
%  \r&=\frac{f}{\cos\phi} \hat i -\vrho 
\end{align}
\subsection{Refraction through oil objective}
\bild{objective.jpg}
\begin{align}
  \a &= f (n-1) \hat z \\
  R &= nf
\end{align}
this is an approximation for small angles: 
\begin{align}
  %\s &= (R - \sqrt{R^2-\rho^2})\i\\
  \r_0 &= \r + \a - \s
\end{align}
\subsection{Refraction at plane surface}
\bild{slab.jpg}
\begin{align}
  k_0&=2\pi/\lambda\\
  k_1&=n_1 k_0\\
  k_2&=n_2 k_0
\end{align}
The normal $\vect{n}$ is directed in the opposite direction of the incoming
wave vector $\k_1$, as you would define it for a mirror. We define the
transversal and normal component vectors (they are perpendicular). FIXME the formulas below and the image don't reflect this. Look at the figure from ``Nomalized results'' instead.
\begin{align}
  \k_{1n}&=(\k_1\n)\n\\ 
  \k_{1t}&=\k_1 - \k_{1n}
\end{align}
the transversal component of the wave vector is maintained:
\begin{align}
  \k_{2t}&=\k_{1t}\\
  k_2^2&=k_{2n}^2 + k_{2t}^2
\end{align}

\subsubsection{Result in terms of wave vector}
\begin{align}
  q &= 1-\k_1 \n \eta = n_2/n_1\\
  \k_2 &= q \k_1 + \sqrt{q^2-\eta^2} k_1\n
\end{align}
\subsubsection{Normalized results}
Its more useful to express all directions with unit vectors
(cg-principles.djvu foley addison-wesley 1990). See the figure.
It also contains the case of total internal reflection.
 \begin{figure} 
   \centering
   %\def\svgwidth{\columnwidth} % sets the image width, this is optional
   \input{refl2.pdf_tex}
 \end{figure}

\end{document}
