% first i tried to use org mode, but that doesn't have sufficient
% support for formulas and images
%\documentclass[DIV19]{scrartcl}
%\usepackage[paper size={90mm, 120mm},left=2mm,right=2mm,top=2mm,bottom=2mm,nohead]{geometry}

\documentclass[twocolumn,DIV19]{scrartcl}
\usepackage[utf8]{inputenc}
%\usepackage[T1]{fontenc}
\usepackage{graphicx}
\usepackage{longtable}
\usepackage{float}
\usepackage{wrapfig}
\usepackage{soul}
\usepackage{amssymb}
\usepackage{amsmath}
\usepackage{hyperref}
\usepackage{color}
\usepackage{units}

\title{Raytracing through an oil objective and non-matching embedding
  medium} 
\author{Martin Kielhorn}
\date{2011-04-11}

\DeclareMathOperator{\sign}{sign}

\begin{document}

\maketitle

\setcounter{tocdepth}{3}
\tableofcontents
\vspace*{1cm}

\newcommand{\vect}[1]{\mathbf{#1}}
\renewcommand{\r}{\vect r}
\renewcommand{\a}{\vect a}
\newcommand{\s}{\vect s}
\def\k{\vect k}
\def\d{\vect d}
\def\e{\vect e}
\def\f{\vect f}
\def\c{\vect c}
\def\x{\vect x}
\def\y{\vect y}
\def\z{\vect z}
\def\q{\vect q}
\def\p{\vect p}
\def\l{\vect l}

\newcommand{\nvect}[1]{\vect{\hat{#1}}}
\renewcommand{\i}{\nvect i}
\def\hc{\nvect c}
\def\hs{\nvect s}
\def\hd{\nvect d}
\def\hx{\nvect x}
\def\hy{\nvect y}

\def\hz{\nvect z}
\def\n{\nvect n}
\def\t{\nvect t}
\def\m{\nvect m}
\def\vrho{\boldsymbol\rho}
\def\abs#1{\mathopen| #1 \mathclose|}

\newcommand{\bild}[1]{\includegraphics[width=7cm]{#1}}
\section{Refraction}
\subsection{Refraction at plane surface}
\begin{align}
  k_0&=2\pi/\lambda\\
  k_1&=n_1 k_0\\
  k_2&=n_2 k_0
\end{align}
The normal $\n$ is directed in the opposite direction of the incident
wave vector $\k_1$, as you would define it for a mirror. We define the
transversal and normal component vectors.
\begin{align}
  \k_{1n}&=(\k_1\n)\n\\ 
  \k_{1t}&=\k_1 - \k_{1n}
\end{align}
During refraction the transversal component of the wave vector doesn't
change furthermore the normal and transversal component are
perpendicular.
\begin{align}
  \k_{2t}&=\k_{1t}\\
  k_2^2&=k_{2n}^2 + k_{2t}^2
\end{align}
Find the length of the normal component of the outgoing wave vector $\k_2$:
\begin{align}
  k_2^2&=k_{2n}^2 + (\k_1 - \k_{1n})^2\\
  k_{2n}^2&=k_2^2-(\k_1-(\k_1\n)\n)^2\\
  &= k_2^2-(k_1^2-2(\k_1\n)^2+(\k_1\n)^2)\\
  &= k_2^2-k_1^2+(\k_1\n)^2
\end{align}
\begin{align}
  \k_2&=\k_{1t}-\sqrt{k_2^2-k_1^2+(\k_1\n)^2}\n\\
  &=\k_1-(\k_1\n)\n-\sqrt{k_2^2-k_1^2+(\k_1\n)^2}\n
\end{align}
Divide by $k_2$ with $\k_2/k_2=\t$ and $\k_1/k_2=\eta\i$ in order to
introduce unit direction vectors $\i$ and $\t$ for incident and
outgoing light. The relative index change across the interface is
$\eta=n_1/n_2$.
\begin{align}
  \t&=\eta\i-\eta(\i\n)\n-\sqrt{1-\eta^2+\eta^2(\i\n)^2}\n\\
  &=\boxed{\eta\i-\left(\eta\i\n+\sqrt{1-\eta^2(1-(\i\n)^2)}\right)\n}
\end{align}
 \begin{figure}
   \centering
   %\def\svgwidth{\columnwidth} % sets the image width, this is optional
   \input{refraction.pdf_tex}
   \caption{Refraction at an interface transforms the incident wave
     vector $\k_1$ into the outgoing wave vector $\k_2$.}
 \end{figure}
 When the radical in the square root is negative a reflection occurs
 instead. The tangential component stays and normal component inverts
 the sign:
 \begin{align}
   \k_2&=\k_{1t}-\k_{1n}\\
   &=\k_1 - 2\k_{1n}\\
   &=\k_1-2(\k_1\n)\n\\
   \t&=\boxed{\i-2(\i\n)\n}
 \end{align}
\subsection{Intersection of a ray and a plane}
A ray may start at a point $\s$ and may have the direction $\hd$.  A
plane that contains the point $\c$ and has the normal $\n$ intersects
with this ray if the normal and ray direction are not perpendicular
$\n\hd\not=0$. The distance between the plane and the origin is
$h=\c\n$. We can define the plane equation in Hesse normal form:  
\begin{align}
  \r\n=h
\end{align}
Now we replace the coordinate $\r$ with the ray equation and solve for
the parameter $\tau$:
\begin{align}
  (\s+\tau\hd)\n&=h\\
  \s\n+\tau\hd\n&=h\\
  \tau&=\boxed{\frac{h-\s\n}{\hd\n}}
\end{align}
 \begin{figure}[!hbt]
   \centering
   %\def\svgwidth{\columnwidth} % sets the image width, this is optional
   \input{plane-intersection.pdf_tex}
   \caption{Schematic for describing the plane-ray intersection.}
 \end{figure}
\subsection{Intersection of a ray and a sphere}
A ray may start at a point $\s$ and may have the direction $\hd$.  A
sphere is defined by its center $\c$ and its radius $R$.  The two
equations
\begin{align}
  (\r-\c)^2&=R^2\\
  \r&=\s+\tau\hd
\end{align}
define the intersection points. Substitution of $\r$ results in a
quadratic equation for $\tau$:
\begin{align}
  (\s+\tau\hd-\c)^2&=R^2\\
  \l&:=\boxed{\s-\c}\\
  l^2+2\tau\l\hd+\tau^2-R^2&=0\\
  \tau^2+\underbrace{2\l\hd}_b\tau+\underbrace{l^2-R^2}_c&=0
\end{align}
\subsubsection{Solving the quadratic equation}
If the determinant $d$ is negative the ray misses the sphere and there
is no solution. If the determinant is zero the ray touches the
periphery and there is only one solution. A positive determinant
corresponds to two solutions.
\begin{align}
  d&:=\boxed{b^2-4ac}\\
  q&:=\boxed{\frac{b+\sqrt{d}\sign b}{2}}\\
  \tau&=\boxed{
  \begin{cases}
    \frac{q}{a} &\,\textrm{when}\,\abs{q}\approx 0\\ 
    \frac{c}{q} &\,\textrm{when}\,\abs{a}\approx 0\\
    (\frac{q}{a}, \frac{c}{q}) &\,\textrm{else}
  \end{cases}}
\end{align}
\subsection{Refraction on thin lens}
%\bild{thin-lens.jpg}
\begin{figure}[!hbt]
  \centering
  % \def\svgwidth{\columnwidth} % sets the image width, this is optional
  \input{lens-fwd.pdf_tex}
  \caption{Construction of a ray on a thin lens.}
\end{figure}
\begin{itemize}
\item calculate intersection $\vrho$ of incident ray which is defined
  by its direction $\i$ and some start position
\item draw a line with direction $\i$ through the center of the lens,
  its intersection with the focal plane defines where the direction
  $\r$ of the exiting ray
\item the length of the hypothenuse $\overline{OF}$ is $f/\cos\theta$
\item triangle $ABC$ is similar to triangle $FOA$, therefor we can
  obtain the length $\abs{\overline{BC}}$ 
  \begin{align}
    \frac{\abs{\overline{BC}}}{\overline{BA}}&=
    \frac{\abs{\overline{OA}}}{\overline{OF}}\\
    \frac{\abs{\overline{BC}}}{1}&=
    \frac{\rho}{f/cos\theta}
  \end{align}
\item the result can be used to calculate the direction $\r'$
\item using the two similar triangles again, $\r'$ can be scaled to
  obtain $\r$
\end{itemize}
(See 2008Hwang)
\begin{align}
  \vrho&=(x_0,y_0,0)^T=\rho (\cos\phi,\sin\phi,0)^T\\
  \phi&=\boxed{\arctan(y_0/x_0)}\\
  \cos\theta&=\boxed{\i\hz}\\
  \r'&=\i- \frac{\cos\theta}{f}\vrho\\
   \r&=\boxed{\frac{f}{\cos\theta} \i -\vrho}
\end{align}
\subsection{Refraction through oil objective (from BFP towards FFP)}
%\bild{objective.jpg}
\begin{figure}[!hbt]
  \centering
  % \def\svgwidth{\columnwidth} % sets the image width, this is optional
  \input{obj-fwd.pdf_tex}
  \caption{Construction of a ray on a thin lens.}
\end{figure}
\begin{align}
  \a &= \boxed{f (n-1) \hz} \\
  R &= \boxed{nf}
\end{align}
\begin{itemize}
\item take into account immersion medium by shifting focal plane to
  $nf$
\item for high NA objectives rays don't originate from a ``plane''
  principal plane but from the gaussian sphere of radius $nf$
\item express the deviation from the principal plane and gaussian
  sphere with an approximation for small angles $\theta$ and $\phi$
\begin{align}
  \s &= \boxed{(R - \sqrt{R^2-\rho^2})\i}
\end{align}
\item this is an approximation as it only takes into account the
  perpendicular distance between plane and sphere
\item 2008whang compares this approximation with an exact raytrace
  through a real $100\times\,1.41$ objective, focus displacement
  errors are less than \unit[130]{nm} for a field of $\unit[86.4]{\mu
    m}$ radius
\item the final ray exiting the objective has the direction $\r_0$
\begin{align}
  \r_0 &= \boxed{\r + \a - \s}
\end{align}
\end{itemize}
\subsection{Reverse path through oil objective}
\begin{itemize}
\item draw line at center of lens
\item mark left and right of that line the (non-immersion) focal
  planes in $f$ distance
\item on the left draw the gaussian circle of radius $nf$ that touches
  the center of the lens on the optical axis
\item draw the real focal plane in distance $nf$ from center of the
  lens
\item start ray anywhere on the left side
\item extend the ray to hit the real front focal plane as well as the
  gaussian circle
\item translate the intersection between ray and real front focal
  plane along optical axis onto the non-immersion focal plane
\item draw a line from this point on the non-immersion focal plane
  through the center of the lens
\item this ray isn't influenced by the lens and defines the angle of
  the other ray as well
\end{itemize}
 \begin{figure}[!hbt]
   \centering
   \input{obj-rev-full.pdf_tex}
   \caption{Construction to find the outgoing ray for a point from the
     immersed side.}
 \end{figure}
\subsubsection{Treatment of aberration}
\begin{itemize}
\item given a ray with starting point $\p$ and direction $\i$
\item find intersection $\f$ of ray with interface and refract to
  obtain $\i'$
\item calculate the duration $t$ a photon would travel from $\p$ to
  the interface $\f$: $t=\abs{\f-\p}n_ec$
\item extend path of the photon backward along $\i'$ by $t/(cn)$, this
  results in the corrected position $\p'$
\item apply the method of the previous section to obtain the ray
  leaving the objective
\end{itemize}
 \begin{figure}[!hbt]
   \centering
   \input{obj-rev-full-emb.pdf_tex}
   \caption{Construction that treats the interface between embedding
     and immersion medium}
 \end{figure}

\section{Tracing from sample space into the back focal plane}
\subsection{Unaberrated objective}
A unit ray direction $\i=(x,y,z)^T$ in sample space is transformed
into a position $\r_b=(x',y')^T$ in the back focal plane of the
objective. This result was first produced in D6.8a.  The azimuthal
angle $\phi$ isn't changed when going through the objective. The angle
$\theta$ defines how far off axis the back focal plane is hit.
\begin{align}
  \phi'&=\phi=\arctan(y/x)\\
  \theta&=\arcsin(\sqrt{x'^2+y'^2})\\
  r_b&=nf\sin\theta\\
  \r_b&=r_b(\cos\phi',\sin\phi')^T
\end{align}
 \begin{figure}[!hbt]
   \centering
   %\def\svgwidth{\columnwidth} % sets the image width, this is optional
   \input{obj-rev.pdf_tex}
   \caption{Schematic for tracing a ray direction $\i$ from sample
     space into the back focal plane. The bigger the angle between
     $\i$ and the optical axis, the further outside the ray will pass
     through the back focal plane.}
 \end{figure}
 We are not interested, in what angle the back focal plane is
 hit. Therefore we don't have to consider the object position.
\subsection{Aberrated objective}
We can use the formulas from above to trace through an embedding
medium of non-matching refractive index $n_e$. Then the ray direction
$\i$ is to be taken after the ray was refracted through the interface
between the embedding medium and the immersion oil. In this case the
position on the back focal plane will depend on sample position.
 \begin{figure}[!hbt]
   \centering
   %\def\svgwidth{\columnwidth} % sets the image width, this is optional
   \input{obj-rev-emb.pdf_tex}
   \caption{Schematic for tracing a ray from inside an embedding
     medium that has a different refractive index than the immersion
     oil.}
 \end{figure}
\section{Nucleus projection}
\begin{itemize}
\item project each out of focus nucleus through each in-focus nucleus 
\item estimate where out of focus nuclei will end up in the BFP
\item the in focus target nucleus is sampled in several points
\item such a target point $\c$ and an out of focus nucleus define a
  double-cone bound by the tangents
\item we call the center of the out of focus nucleus $\s$
\item construct the tangents on the out of focus nucleus by
  intersecting a sphere around $\c$ with radius $R=\abs{\c-\s}$ and
  the sphere of the out of focus nucleus centered at $\s$ with radius
  $r$
\item the intersection point is $\e$
\end{itemize}

\begin{figure}[!hbt]
  \centering
  \input{touch-cone.pdf_tex}
  \caption{Schematic of how an out of focus nucleus defines a cone of
    rays.}
\end{figure}
It is sufficient to solve the following equation in a 2D coordinate
system with the origin in the center of the out of focus nucleus:
\begin{align}
  (x-R)^2+y^2&=R^2\\
  x^2+y^2=r^2
\end{align}
Leading to two solutions:
\begin{align}
  x_1&=\frac{r^2}{2R}\\
  y_{1/2}&=\pm\frac{r}{2R}\sqrt{4R^2-r^2}
\end{align}
In the case $R<r$ the out of focus nucleus is intersecting the target,
obliviating the reason to do the projection in the first place.

The (unnormalized) direction $\x$ of the x-axis of this coordinate
system is given by the difference vector of the target $\c$ and the
nucleus center $\s$. The direction $\y$ must be perpendicular to $\x$
and is obtained by calculating the cross product with another vector
$\q$.  It must be ensured that $\q$ isn't colinear with $\x$. The
vectors $\q$ and $\x$ are colinear, when the absolute value of their
scalar product equals the square of the length $\abs{\q\x}=\x^2$.
\begin{align}
  \x&=\c-\s\\
  \q&=\begin{cases}
    (0,0,1)^T & \textrm{when}\ \abs{x_z}<\frac{2}{3}\abs{\x}\\
    (0,1,0)^T & \textrm{else}
  \end{cases}\\
  \y&=\x\times\q \\
  \hx&=\x/\abs{\x}\\
  \hy&=\y/\abs{\y}
\end{align}
\subsection{Generate points on the out of focus nucleus}
\begin{itemize}
\item the cone defines a bill board circle
\item we want to sample this circle
\end{itemize}
Let $R_\phi^\hc$ be a rotation matrix that rotates a vector by angle
$\phi$ around an axis $\hc$. This we can use to obtain a point $\e$
on the bill board circle. Then we can find a ray direction $\f$ on the
cone.
\begin{align}
  \e&=\s+x_1\hx+y_1R_\phi^\hx\hy\\
  \f&=\c-\e
\end{align}
Tracing a sufficient number of rays (e.g.\ 16) with direction $\f$ for
different angles $\phi$ to the back focal plane gives the projection
of the bill board.

For practical reasons its useful to project vector $\x$ as well. It
can be used as the center of the (distorted) shape on the back focal
plane to render it as a fan of triangles.
\subsection{Test}
Set the target point $\c=(0,0,0)^T$ and the nucleus center to
$\s=\unit[(6,0,7)^T]{n_e \mu m}$ with a nucleus radius
$r=\unit[1.2]{n_e \mu m}$. Note that dimensions are multiplied with
the refractive index $n_e=1.33$ of the embedding medium (water) as for
the light the medium increases distances.

\section{Ferroelectric liquid crystals}
(See 1991Saleh p.~722 and Goodman 192)

\section{Simple sketch of MEMI system}
\begin{figure}[!hbt]
  \centering
  % \def\svgwidth{\columnwidth} % sets the image width, this is optional
  \input{memi-simple.pdf_tex}
  \caption{Simplified schematic of the MEMI system.}
\end{figure}
\end{document}

