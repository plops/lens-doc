% first i tried to use org mode, but that doesn't have sufficient
% support for formulas and images
\documentclass[twocolumn,DIV19]{scrartcl}
\usepackage[utf8]{inputenc}
%\usepackage[T1]{fontenc}
\usepackage{graphicx}
\usepackage{longtable}
\usepackage{float}
\usepackage{wrapfig}
\usepackage{soul}
\usepackage{amssymb}
\usepackage{amsmath}
\usepackage{hyperref}
\usepackage{color}

\title{aberration}
\author{martin}
\date{31 March 2011}

\begin{document}

\maketitle

\setcounter{tocdepth}{3}
\tableofcontents
\vspace*{1cm}

\newcommand{\vect}[1]{\mathbf{#1}}
\renewcommand{\r}{\vect r}
\renewcommand{\a}{\vect a}
\newcommand{\s}{\vect s}
\def\k{\vect k}
\def\d{\vect d}

\newcommand{\nvect}[1]{\vect{\hat{#1}}}
\renewcommand{\i}{\nvect i}
\newcommand{\z}{\nvect z}
\def\n{\nvect n}
\def\t{\nvect t}
\def\m{\nvect m}
\def\vrho{\boldsymbol\rho}
\def\abs#1{\mathopen| #1 \mathclose|}
\newcommand{\bild}[1]{\includegraphics[width=7cm]{#1}}
\section{Aberration due to water}
\subsection{Refraction on thin lens}
\bild{thin-lens.jpg}

We define the intersection of the incident ray
with the XY-plane of the lens as $\vrho$.
\begin{align}
  \vrho&=(x_0,y_0,0)^T=\rho e^{i\phi}\\
  \phi&=\arctan(y_0/x_0)\\
  \cos\theta&=\i\z\\
  \r'&=\i- \frac{\cos\theta}{f}\vrho\\
  \r&=\frac{f}{\cos\theta} \i -\vrho 
\end{align}
\subsection{Refraction through oil objective}
\bild{objective.jpg}
\begin{align}
  \a &= f (n-1) \z \\
  R &= nf
\end{align}
We express the deviation from the principal plane and gaussian sphere
with an approximation for small angles:
\begin{align}
  \s &= (R - \sqrt{R^2-\rho^2})\i\\
  \r_0 &= \r + \a - \s
\end{align}
\subsection{Refraction at plane surface}
\bild{slab.jpg}
\begin{align}
  k_0&=2\pi/\lambda\\
  k_1&=n_1 k_0\\
  k_2&=n_2 k_0
\end{align}
The normal $\n$ is directed in the opposite direction of the incident
wave vector $\k_1$, as you would define it for a mirror. We define the
transversal and normal component vectors.
\begin{align}
  \k_{1n}&=(\k_1\n)\n\\ 
  \k_{1t}&=\k_1 - \k_{1n}
\end{align}
During refraction the transversal component of the wave vector doesn't
change furthermore the normal and transversal component are
perpendicular.
\begin{align}
  \k_{2t}&=\k_{1t}\\
  k_2^2&=k_{2n}^2 + k_{2t}^2
\end{align}
Find the length of the normal component of the outgoing wave vector $\k_2$:
\begin{align}
  k_2^2&=k_{2n}^2 + (\k_1 - \k_{1n})^2\\
  k_{2n}^2&=k_2^2-(\k_1-(\k_1\n)\n)^2\\
  &= k_2^2-(k_1^2-2(\k_1\n)^2+(\k_1\n)^2)\\
  &= k_2^2-k_1^2+(\k_1\n)^2
\end{align}
\begin{align}
  \k_2&=\k_{1t}-\sqrt{k_2^2-k_1^2+(\k_1\n)^2}\n\\
  &=\k_1-(\k_1\n)\n-\sqrt{k_2^2-k_1^2+(\k_1\n)^2}\n
\end{align}
Divide by $k_2$ with $\k_2/k_2=\t$ and $\k_1/k_2=\eta\i$ in order to
introduce unit direction vectors $\i$ and $\t$ for incident and
outgoing light. The relative index change across the interface is
$\eta=n_1/n_2$.
\begin{align}
  \t&=\eta\i-\eta(\i\n)\n-\sqrt{1-\eta^2+\eta^2(\i\n)^2}\n\\
  &=\eta\i-\left(\eta\i\n+\sqrt{1-\eta^2(1-(\i\n)^2)}\right)\n
\end{align}
 \begin{figure}
   \centering
   %\def\svgwidth{\columnwidth} % sets the image width, this is optional
   \input{refraction.pdf_tex}
   \caption{Refraction at an interface transforms the incident wave
     vector $\k_1$ into the outgoing wave vector $\k_2$.}
 \end{figure}
 When the radical in the square root is negative a reflection occurs
 instead. The tangential component stays and normal component inverts
 the sign:
 \begin{align}
   \k_2&=\k_{1t}-\k_{1n}\\
   &=\k_1 - 2\k_{1n}\\
   &=\k_1-2(\k_1\n)\n\\
   \t&=\i-2(\i\n)\n
 \end{align}
\section{Tracing from sample space into object space}
\subsection{Unaberrated objective}
A unit ray direction $\i=(x,y,z)^T$ in sample space is transformed
into a position $\r_b=(x',y')^T$ in the back focal plane of the
objective. This result was first produced in D6.8a.  The azimuthal
angle $\phi$ isn't changed when going through the objective. The angle
$\theta$ defines how far off axis the back focal plane is hit.
\begin{align}
  \phi'&=\phi=\arctan(y/x)\\
  \theta&=\arcsin(\sqrt{x'^2+y'^2})\\
  r_b&=nf\sin\theta\\
  \r_b&=r_b(\cos\phi',\sin\phi')^T
\end{align}
 \begin{figure}[!hbt]
   \centering
   %\def\svgwidth{\columnwidth} % sets the image width, this is optional
   \input{obj-rev.pdf_tex}
   \caption{Schematic for tracing a ray direction $\i$ from sample
     space into the back focal plane. The bigger the angle between
     $\i$ and the optical axis, the further outside the ray will pass
     through the back focal plane.}
 \end{figure}
 We are not interested, in what angle the back focal plane is
 hit. Therefore we don't have to consider the object position.
\subsection{Aberrated objective}
We can use the formulas from above to trace through an embedding
medium of non-matching refractive index $n_e$. Then the ray direction
$\i$ is to be taken after the ray was refracted through the interface
between the embedding medium and the immersion oil. In this case the
position on the back focal plane will depend on sample position.
 \begin{figure}[!hbt]
   \centering
   %\def\svgwidth{\columnwidth} % sets the image width, this is optional
   \input{obj-rev-emb.pdf_tex}
   \caption{Schematic for tracing a ray from inside an embedding
     medium that has a different refractive index than the immersion
     oil.}
 \end{figure}

\end{document}
